Abstracts must be strictly between 1500 and 2000 characters and are in pure text format, copied to a dialog box in the conference management system. Only common latex/tex markup code is acceptable, but this will be counted as characters (AMS packages special characters accepted).

To submit an abstract you need to provide the following information:
Title of the submission
Tagging according to math/computer science methodology on Topics and Sessions
Tagging according to geoscience field on Topics and Sessions
If a devoted session already exists, submission to that particular session. These will be announced in due course. Session-free contributions are also possible, and encouraged.
Names, e-mail addresses and institutions of all authors.
Presenting author
A 1500 to 2000 character abstract. This can be copy-pasted from a plain text or simple tex/latex document of your own.


Model based clustering applied to geosciences

 crystal chemistry with the composition of minerals characterised by mixing starting from defined end-members

very large data base with minerals from several parts of the Earth or time


Finite mixtures of distributions are the most common distributions to model different sources of variability within a dataset. Although the finite mixture of Gaussian distributions is the most widely used, it exhibits some restrictions when the sources of variability have different nature. New techniques have proposed to model each source of variability with a different mixture of Gaussian distributions, and therefore, to model the full dataset using a finite mixture of Gaussian mixture distributions.

