\documentclass[a4paper,11pt]{article}

\title{}
\author{}

\begin{document}
\maketitle

Finite mixtures of distributions are the most common distributions used to model different sources of variability within a data set. Because of this, finite mixtures of distributions are the most suitable models to do model-based clustering. Although the advantages of using finite mixtures models, they exhibit some restrictions when the different sources of variability have different nature. To deal with this restriction, different approaches have been proposed to combine different components into one component which models a single source of variability. The result is that the full data set is modelled by a finite mixture of finite mixtures distributions.

In this presentation, we use a real data set coming from (DATA SET'S AREA) to introduce and compare the commented approaches. In this work, we always take into an account sample space of our data .i.e the Simplex. We analyse our data set by using finite mixture of Gaussian distributions. Then, we consider different approaches to combine those components which are more likely to form a single group, and therefore, to generate a single source of variability. To compare the groups obtained by the different approaches, we propose a graphical tool to visualise the similarity between the obtained clusters.

The analysed data set consist of (DATA SET EXPLANATION).

\end{document}


