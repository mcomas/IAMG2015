Abstracts must be strictly between 1500 and 2000 characters and are in pure text format, copied to a dialog box in the conference management system. Only common latex/tex markup code is acceptable, but this will be counted as characters (AMS packages special characters accepted).

To submit an abstract you need to provide the following information:
Title of the submission
Tagging according to math/computer science methodology on Topics and Sessions
Tagging according to geoscience field on Topics and Sessions
If a devoted session already exists, submission to that particular session. These will be announced in due course. Session-free contributions are also possible, and encouraged.
Names, e-mail addresses and institutions of all authors.
Presenting author
A 1500 to 2000 character abstract. This can be copy-pasted from a plain text or simple tex/latex document of your own.


Model based clustering applied to geosciences

 crystal chemistry with the composition of minerals characterised by mixing starting from defined end-members

very large data base with minerals from several parts of the Earth or time

More general distributions have been proposed to model each source of variability (skew-normal distribution, skew-t distributions or the hypergeometric distribution).

######

Finite mixtures of distributions are the most common distributions used to model different sources of variability within a dataset. Although finite mixtures of Gaussian distributions are the most widely used, they exhibit some restrictions when the sources of variability have different nature. To deal with this restriction, different approaches have been proposed to combine different components into one component which models a single source of variability. The result is that the full dataset is modelled by a finite mixture of finite mixtures distributions.

In this presentation, we use a real dataset coming from (DATASET'S AREA) to introduce and compare the commented approaches. We always consider the compositional sample space of our data, we analyse our dataset by using finite mixture of Gaussian distributions defined on the Simplex. Then, we consider different approaches to combine those components which are more likely to form single group, and therefore, to generate a single source of variability. To compare the groups obtained by the different approaches, we propose a graphical tool to visualise the similarity between the obtained clusters.

The analysed dataset consist of (DATASET EXPLANATION).

