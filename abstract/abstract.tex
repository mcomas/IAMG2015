\documentclass[a4paper,11pt]{article}
\usepackage[utf8]{inputenc}

\title{Finite mixtures of distributions: compositional model-based clustering}
\author{Marc Comas-Cufí \and Antonella Buccianti \and Glòria Mateu-Figueras \and Josep-Antoni Martín-Fernández}

\begin{document}
\maketitle

Finite mixture of distributions is a most common tool to model different sources of variability within a data set. When it is applied to make groups of samples, it is usually referred as model-based clustering. Finite mixtures models exhibit some limitations when the sources of variability have very different features. To deal with these difficulties, one considers that each cluster is a single source of variability. In consequence, a cluster can be formed by the combination of different mixture components.

In this contribution, we use a real data set coming from geochemistry to introduce and compare different approaches and illustrate their applications. Because of the nature of our data set, we apply compositional analysis techniques to take into account the geometry of the sample space. In particular, we model the data set by using finite mixture of Gaussian distributions on the simplex defined on log-ratio coordinates. When combining mixture components to construct the clusters, different approaches based on the posterior probabilities are evaluated. One of these approaches is a novel technique based on log-ratios. To visualise the similarity between the resulting groups a new graphical tool is introduced.

The data set used to test the methodology is given by the chemistry of the groundwaters of more than 4000 samples collected in Tuscany region (central Italy). Hydrochemical facies is a term used to denote the diagnostic chemical aspect of groundwater solutions occurring in hydrologic systems. The facies reflect the effects of chemical processes occurring between the minerals within the lithologic framework and the groundwater. The flow patterns modify the facies and control their distribution. The distribution of these facies, as well as the relations between rock types and groundwater composition are commonly displayed in trilinear diagrams, as for example the Piper plot. The evaluation of the facies is thus only performed on a descriptive way.

\end{document}


